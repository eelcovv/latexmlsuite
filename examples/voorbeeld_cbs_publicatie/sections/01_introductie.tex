\chapter{Introductie}
\label{ch:introductie}


\section{Voorbeelden die besproken worden}
Dit document kan gebruikt worden als opstart document voor een CBS-publicatie die je aan CCN kan
aanleveren.

In hoofdstuk~\ref{ch:voorbeelden}  zullen wat elementen van \LaTeX laten zien.
In sectie~\ref{sec:basis_environment} laten we bullet lijsten zien.
Hoe je een highcharts of python plaatje kan toevoegen wordt in sectie~\ref{sec:plaatje_toevoegen}
laten zien.
Het highcharts plaatje komt in sectie~\ref{subsec:plaatje_highcharts} aan bod, terwijl we het
plaatje gemaakt met Python in sectie~\ref{subsec:plaatje_python} behandelen.

We kunnen ook tabellen in CBS-formaat toevoegen, wat in sectie~\ref{sec:tabel} besproken wordt.

Ten slotte laten we in de appendix~\ref{ch:extra} zien hoe je met Makefiles werkt.

De~\href{https://github.cbsp.nl/EVLT/cybersecuritymonitor-2020}{Cybersecuritymonitor 2020}
\citep{cybersecuritymonitor_20} is de eerste officiële CBS-publicatie waarbij
deze conversie methode vanuit \LaTeX toegepast is.
De broncode staat ook op de intern~\href{https://github/evlt/cybersecuritymonitor}{CBS github}
\citep{cybersecuritymonitor_20_github}.

De conversie van latex naar html wordt gedaan met het LatexXML~\citep{latexml}, een open-source
software tool.
De html-bestanden die \emph{latexml} genereert, proberen zo nauwkeurig mogelijk het oorspronkelijk
\LaTeX document na te bootsen, zodat er allerlei stijlelementen aan de html's toegevoegd worden.
Voor de levering aan CCN is het juist nodig om een zo'n leeg mogelijk html aan te leveren.
Het opschonen van de html-bestanden gaat met
\emph{latexml2sitescore}~\citep{latexmlsitescore_github}, een intern ontwikkelde Python tool.

Als je nog nieuw bent met Latex: meer achtergrondinformatie kan je vinden in
\emph{The Not So Short Introduction to \LaTeX}~\citep{latexnottooshort}.



